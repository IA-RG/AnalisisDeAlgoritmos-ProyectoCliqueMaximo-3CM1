%% Las secciones del "prefacio" inician con el comando \prefacesection{T'itulo}
%% Este tipo de secciones *no* van numeradas, pero s'i aparecen en el 'indice.
%%
%% Si quieres agregar una secci'on que no vaya n'umerada y que *tampoco*
%% aparesca en el 'indice, usa entonces el comando \chapter*{T'itulo}
%%
%% Recuerda que aqu'i ya puedes escribir acentos como: 'a, 'e, 'i, etc.
%% La letra n con tilde es: 'n.

\prefacesection{Introducci'on}

El  problema  del  clique  máximo  es  un  problema  de  optimización  combinatoria  que  se  clasifica dentro de los problemas NP- duros los cuales son difíciles de resolver. Debido a su complejidad las técnicas convencionales exactas (exhaustivas) tardan mucho tiempo para dar una solución, por lo tanto es necesario desarrollar algoritmos heurísticos que lo resuelvan  alcanzando  una  solución  cercana  al  óptimo  en  un  tiempo  razonable.  Este  problema tiene aplicaciones reales como son: teoría de códigos, diagnóstico de errores, visión   computacional,   análisis   de   agrupamiento,   recuperación   de   información,   aprendizaje automático, minería de datos, entre  otras.    Por  esta  razón  es  importante  usar  nuevas técnicas heurísticas y/o metaheurísticas para tratar de resolver este problema, las cuales obtengan mejores resultados en un tiempo polinomial. 



%% Por si alguien tiene curiosidad, este "simp'atico" agradecimiento est'a
%% tomado de la "Tesis de Lydia Chalmers" basada en el universo del programa
%% de televisi'on Buffy, la Cazadora de Vampiros.
%% http://www.buffy-cazavampiros.com/Spiketesis/tesis.inicio.htm
